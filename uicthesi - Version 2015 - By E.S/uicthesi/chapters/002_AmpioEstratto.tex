\estratto

\label{chap:ampio_estratto}

Questa tesi presenta due contributi distinti e complementari. Il primo contributo é uno strumento software chiamato RoomPlaces che fornisce un sistema di localizzazione di prossimità che sfrutta la tecnologia Bluetooth Low Energy. Il secondo contributo è uno studio esplorativo riguardante l’utilizzo di artefatti tangibili a scopo didattico condotto utilizzando RoomPlaces per creare WallCology (Moher, et al., 2008): un’istanza di Embedded Phenomena che simula il funzionamento di un ecosistema all’interno di una classe di scuola primaria.

La creazione di artefatti tangibili e il loro utilizzo all’interno di applicazioni didattiche é una possibilità che merita di essere esplorata e attorno alla quale sono state costruite le domande alle quali lo studio esplorativo ha l’obiettivo di rispondere. In letteratura ci sono molte prove che mettono in relazione il movimento fisico degli utenti durante le attività didattiche con una serie di benefici per l’apprendimento quali aumento della concentrazione, migliore capacità di ricordare i concetti importanti e non ultimo il divertimento nell’usare la tecnologia per svolgere i compiti assegnati in un ambiente che favorisce le interazioni sociali e lo scambio di idee. L’utilizzo di interfacce tangibili al posto delle consuete interfacce grafiche é altrettanto supportato dalla letteratura, in quanto permette un uso estensivo del movimento del corpo come mezzo di comunicazione con il mondo digitale della simulazione con tutti i benefici precedentemente indicati. Inoltre queste interfacce, se utilizzate nel modo corretto, possono produttivamente stimolare discorsi riguardanti gli argomenti didattici in questione e lo scambio di informazioni. Il maggior problema didattico che l’utilizzo di artefatti tangibili si propone di affrontare é la difficoltà da parte degli studenti nel condividere le informazioni durante lo svolgimento delle attività, specialmente se interattive supportate dalla tecnologia informatica. 

Esiste anche un’altra motivazione secondaria che supporta l’utilizzo di interfacce tangibili in certi tipi di applicazione: la creazione e utilizzo di una mini cultura attorno ad esse. Questa idea, rivelatasi non completamente applicabile nello studio condotto, é basata sul riutilizzo di conoscenze già possedute, che può essere sfruttata per favorire la manipolazione di oggetti in un modo conosciuto per ottenere un esito prevedibile (da parte del progettista), al posto di dover imparare come usare nuove interfacce mai viste prima.

RoomPlaces é uno strumento software che consente agli sviluppatori di creare applicazioni che hanno accesso alla posizione di oggetti e persone all’interno della classe e fornisce notifiche quando questi arrivano o lasciano determinate aree. Tra le varie possibilità che il sistema offre si possono trovare una serie di utili servizi per gli insegnanti, ad esempio: appello automatizzato, gestione di oggetti nell’inventario della classe e monitoraggio dell’utilizzo delle risorse. Pur essendo focalizzato su come utilizzare la posizione di oggetti come interfaccia verso simulazioni scientifiche didattiche, RoomPlaces è in grado di localizzare sia oggetti che persone, per questa ragione presenta una serie di caratteristiche che lo rendono uno strumento innovativo in grado di supportare la maggior parte delle simulazioni scientifiche svolte a scopo didattico nelle scuole. RoomPlaces rappresenta il primo passo per la creazione di un framework multi uso per sfruttare la localizzazione di utenti e oggetti all’interno di una stanza (o di multiple stanze). Vuole inoltre essere un approccio pratico alla realizzazione di un sistema in grado di localizzare utenti (che possiedono smarthphones o tablets) utilizzando Bluetooth Low Energy beacons. È anche possibile utilizzare lo schema opposto, dove beacons vengono localizzati usando dispositivi mobili o uno schema ibrido dove dispositivi e beacons sono usati contemporaneamente per localizzare altri dispositivi e altri beacons (anche se con alcune limitazioni). Come ultima parte del contributo apportato alla ricerca, RoomPlaces contiene un sistema distribuito per associare delle informazioni a degli oggetti fisici al fine di facilitare lo sviluppo di applicazioni che necessitano di associare dei parametri a delle risorse.

Lo strumento é integrato in un framework chiamato “nutella”, che ha lo scopo di supportare tutto il ciclo di vita di un “macroworld”: un’evoluzione del concetto di Embedded Phenomena dove elementi chiave di una simulazione scientifica, distribuita su piú componenti, vengono mappati nello spazio fisico della classe, al fine di coinvolgere ogni studente nella formulazione delle ipotesi ed invogliare la sua partecipazione e l’apprendimento. Il framework nutella é stato creato in parallelo con RoomPlaces e per questo motivo i due sistemi sono completamente integrati. RoomPlaces é composto da diversi componenti che comunicano in rete tra di loro sfruttando nutella per ottenere un’astrazione sulla rete, che, unita ad un sistema (sempre integrato in nutella) che permette l’isolamento tra diverse istanze della stessa applicazione, consente di eseguire l’attività in parallelo in diverse classi. Un’altra ragione per voler utilizzare nutella come base per RoomPlaces sono i vantaggi derivanti dall’utilizzo delle tecnologie già integrate in esso, come un sistema di immagazzinamento dati distribuito basato su MongoDB e una serie di librerie di comunicazione basate sul paradigma “push/pull”.

Lo studio esplorativo é stato condotto contemporaneamente in tre classi di scuola primaria, su bambini di 11 o 12 anni, con lo scopo di stabilire l’effettiva utilizzabilità degli artefatti tangibili all’interno delle attività didattiche e a questo scopo ci si è posti due quesiti ai quali é stata data una risposta alla fine dello studio:
Quali meccanismi studenti ed insegnanti usano per coordinarsi durante l’uso di un solo gruppo di artefatti tangibili necessari per svolgere determinate attività?
Possono gli artefatti tangibili motivare gli studenti e creare l’opportunità di discutere riguardo agli argomenti oggetto della lezione? Esiste la possibilità di riutilizzare conoscenza già appresa per favorire l’utilizzo delle interfacce tangibili al posto di dover apprendere come usarle?
Questo studio é stato condotto utilizzando un’applicazione chiamata WallCology sviluppata con lo scopo di trovare risposte a queste domande. Il programma simula diversi ecosistemi con creature, di undici tipi diversi, che vivono al loro interno e interagiscono tra di loro con relazioni di tipo preda-predatore. Le creature sono tutte immaginarie e il loro aspetto é simile a piccoli insetti. Durante il corso delle simulazioni, vengono forniti agli studenti gli strumenti per vedere come le creature interagiscono tra di loro (con una simulazione in computer grafica), conoscere quantitativamente quante sono in ogni habitat in ogni momento e applicare perturbazioni all’ambiente (come l’introduzione di nuove specie o l’estinzione di una specie esistente). Lo scopo per cui gli studenti operano é capire come funziona la catena alimentare all’interno degli habitat basandosi su evidenze o congetture che possono essere verificate con gli strumenti a disposizione e condivise immediatamente per mezzo di un apposito strumento che raccoglie ipotesi e osservazioni della classe. Sono stati costruiti 19 artefatti tangibili: 11 rappresentano le creature, e sono state rese manipolabili stampando i modelli 3D, 4 sono azioni possibili e 4 sono chiavi necessarie per utilizzare la strumentazione. Quando avvicinati ad un ecosistema, questi artefatti contenenti dei Bluetooth LE beacons vengono localizzati e permettono di svolgere azioni nell’ecosistema stesso senza dover utilizzare l’interfaccia grafica.

I risultati dello studio hanno dimostrato che introdurre artefatti tangibili sia positivo per l’apprendimento nelle classi e questa tecnologia sia abbastanza matura per poter essere considerata una valida alternativa alle interfacce grafiche qualora il contesto lo consenta. Gli insegnanti hanno sempre lasciato liberi gli studenti di interagire con gli artefatti, questo ha fatto in modo che si instaurassero naturalmente diversi sistemi per il coordinamento: la creazione di una coda e la creazione di accordi tra gruppi di studenti per l’ordine di utilizzo. Il fatto che ci fossero solo un gruppo di oggetti ha naturalmente individuato un punto di incontro attorno al tavolo centrale della classe (dove gli artefatti erano depositati dopo l’utilizzo) ed esso è diventato un luogo unico dove gli studenti in attesa di avere gli artefatti potevano discutere degli esperimenti che stavano facendo nei loro ecosistemi. Ci sono forti evidenze nei filmati e nei questionari sottoposti agli studenti che questi scambi di informazioni siano stati incoraggiati dal sistema. Inoltre questo tipo di interfaccia é stata molto apprezzata dagli studenti stessi, confermando la maggior capacità di motivarli e mantenerli concentrati sui loro obiettivi creando un ambiente giocoso e produttivo nel quale apprendere.

Per concludere, RoomPlaces ha supportato efficacemente lo studio svolto sugli studenti delle elementari, confermando la necessità di uno strumento del genere per supportare in futuro le applicazioni che necessiteranno la localizzazione. Questo sistema é solo un primo passo verso una soluzione unica e generale, che supporti tutte le opportunità introdotte dalla localizzazione di cose e persone, e per questo in questa tesi sono stati forniti esempi di utilizzi futuri e possibili spunti per altre ricerche.

