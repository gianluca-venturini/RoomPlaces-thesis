\chapter{Conclusion and Future Work}

\label {ch:conclusions}

\section{Summary}
This thesis presents RoomPlaces: a software framework for supporting science inquiry in the Classroom of Things. In the first chapter I inserted the system in the current context of research motivating the introduction of location services and tangible interaction based interfaces inside classrooms explaining the opportunities available and not yet explored. Some of those opportunities are inspired by systems built in the last decade found in literature. Every system is very specific for the kind of the teaching application where it is used and there's no common ground and code reusing between research groups. The result is that indoor location tracking for learning technologies is still far for being mature and I believe that the first step toward its massive usage is creating a general, reusable tracking system able to locate both users and tangible objects inside classrooms. In the second chapter I described all the systems that take advantage of the location of objects and users as a computer interface for inserting input into scientific inquiries. All the example in literature were used for outline the specifications for the general tracking system called RoomPlaces. Proximity tracking is first functionality that needs to be supported in a tracking systems that aim to be used in schools. By design RoomPlaces establish the position of objects (dynamic resources) using other objects (static resources) as reference point. In the third chapter I explained six month of software engineering work behind the development of the final tool where all the capabilities for supporting the pre-discussed opportunities were implemented. The best way for providing evidence that the tool successfully supports real case of classroom activities is to create a meaningful user study with real research questions and answer them using the capabilities of the software. The forth chapter report the complete user study carried out in two months using RoomPlaces for creating the human-computer interaction through tangible interfaces. After the study I reported the results both answering all the pre-discussd research questions and providing evidence that RoomPlaces and nutella allow to prepare a complete and complex user study like WallCology that last months. The great advantage that comes with RoomPlaces is that the software is already tested and stable. As already reported in the previous chapter RoomPlaces was used only for tracking tangibles inside classrooms, but was designed for tracking also tablets and users with wearable sensors into multiple reference systems (e.g. discrete coordinate system). There are many more opportunities for re-using and improving RoomPlaces in the future, I talk about them in the next section.

\section{Future work}

RoomPlaces is the first step in building a general tracking system that supports all the opportunities in location aware applications possible with the introduction of a tracking system. The proximity tracking system was chosen in this first step because it is the simplest and powerful enough for taking advantage of the opportunities present in classrooms (and listed in the introduction chapter). The next step for continuing the development of RoomPlaces is the integration of more tracking technologies inside the tracking layer. This integration will be possible because of the abstraction layer (RoomPlaces APIs) that completely separates the technology of the tracking system and system calls that are available for the high level application development. Among the other possibilities there are the triangulation and trilateration based technologies that can take advantage of radio and ultrasound signals. These technologies will enable more precise location of users and objects inside the room and will allow to create applications that require great precision for working, for example augmented reality systems where a virtual layer (based on the location of certain objects) is overlapped to the physical world.

Another interesting future work is scaling the capabilities of RoomPlaces in order to support many more users and objects and places tracked than now. In the next paragraph I will describe possible scenario where those technology can improve the life of people.

With minor iterative improvements is will be possible to use RoomPlaces in multi-room environments, employing indoor location in entire buildings and tracking users moving across different ares. It is easy to see how applications can be built for many fields way beyond learning technologies. In museums and interactive exhibitions there's the opportunity to provide the user with location dependent content on personal devices and also on shared media (like big screens shared among more users). Tracking patients and visitors in hospitals with the goal of providing indoor navigation information is a priority in research \cite{yang:ibeacon}. RoomPlaces could be employed for this kind of applications without the need of building custom solutions. Reliability of the system is essential during indoor navigation, information must be available at the right time for being useful. Smart homes are also an active area of research where RoomPlaces could be employed. Knowing how many people are present in each room can be the key for creating user friendly "smart spaces" where lights, temperature and security are location dependent with the goal of reducing energy consumption and release the user from trivial tasks (e.g. turning on lights, lowering the temperature while outside). An extension of the multi-room tracking system is a city-wide tracking system, where sensors are employed for providing high precision location beyond GPS position in specific areas in order to provide services to both users and city infrastructures. Possible applications are fruition of customized content on shared displays (e.g. information about travels of passengers near the display), real-time train scheduling based on people waiting at the stop knowing their destination with the goal of minimize the waiting time during changes.

All the described applications could be supported using RoomPlaces as intermediary layer between the tracking technology and the application. The approach that employ a general tracking system I think it will be successful for attacking location tracking problems because the core functions identified in this thesis are common to all those problems. Different systems require different accuracy, for this reason RoomPlaces must be flexible enough for support different tracking methodologies and provide a unified interface that can be employed in many different scenarios. Using a single general system in multiple applications will enable to update it overtime learning from experiences in different fields.