\summary

\label{chap:summary}

This thesis offers two distinct and complementary contributions. First, it introduces a software framework called RoomPlaces that supports proximity location tracking of students and artifacts in classrooms using Bluetooth Low Energy beacon technology. RoomPlaces provides a set of functions that allow developers to both query the system for the location of an object and subscribe to notifications of arrival and departure events from discrete locations. While these capabilities could be used to support a variety of services such as automatic attendance, inventory management, and monitoring of resource utilization, the focus here is on the use of the location as a driver for scientific simulation. 

Second, RoomPlaces is used to support an exploratory study conducted on the embedded phenomena WallCology (Moher, et al., 2008), where the location of shared tangible artifacts is used to control a distributed simulation of ecosystem dynamics. Students are required to work in groups with the goal of discovering the dynamics of the interactions among species situated inside simulated ecosystems. Students have the capability to alter the species composition and population levels present in those virtual environments and observe how they respond to uncontrollable external pressures such as warming, habitat destruction, and the invasion of alien species.

Research in community knowledge construction has shown that getting students to seek and share information with one another is extremely challenging. Students in three sixth-grade classes were provided with a single set of shared control artifacts with embedded Bluetooth LE beacons, creating a bottleneck designed to promote interaction and information sharing among groups. We present evidence supporting the claim that the use of the tangibles raised learners' motivation to participate in the activity, fostered productive disciplinary discourse, and heightened student sense of autonomy and control during the activity.
