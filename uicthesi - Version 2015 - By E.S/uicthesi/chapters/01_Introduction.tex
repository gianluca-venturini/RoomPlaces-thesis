\chapter{Introduction}

\label{chap:introduction}

In the last two decades the development of new computing technologies has radically expanded and became pervasive in every aspect of our life. Those technologies are transforming and expanding the activity structures that can be designed and built for school classrooms with the goal of supporting students and teachers during the process of learning. The \textit{Classrooms of Things} is becoming a reality: more and more computing artifacts are created and connected to the network with the goal of teaching a particular concept. It is important to understand the implications of this technology and to explore how can we take advantage of it for creating the classroom activities of the future. The vision is building cheap and easy to use applications that can be purchased, configured and ran by school teachers that will be in full control of the whole class activity customizing it for the needs of the students.

Past researches in the field of learning sciences highlighted the opportunity of creating a system that address the needs of using objects as interface for interacting with computing devices. In this thesis I explore how to take advantage of this opportunity for integrating embodied and tangible interactions inside a large variety of activities that involve the whole classroom. There are mainly two trends that take a different approach in order to enhance the excitement and stimulate the process of learning through hands-on activities: \textit{embodiment} and \textit{tangibles}. In embodiment the movement of the whole body is used as a way for interacting with the system: it is based on different body metaphors associating different movements and gestures with different concepts that have to be teached. Tangibles are physical objects used for interacting with virtual environments. They can be used as input devices or as computing systems themselves.

\section{Whole body movement}
With the term \textit{embodiment} I mean the enactment of knowledge and concepts though the activities of the whole body. The activities based on that are considered from the learning science community to have a strong potential to engage learners in immersive experiences that enhance their education \cite{lindgren:emboldened}. It is based on the simple idea that human cognition is deeply rooted in the body’s interactions with its physical environment. There are increasing evidence that body movements (for example gestures) can facilitate the retrieval of mental and lexical items. That’s because cognitive processes involved in learning (e.g. conceptual development and comprehension) are built upon a foundation of physical embodiment. When the right sensorimotor systems are engaged a stronger and persistent knowledge representation is created \cite{lindgren:emboldened} and for this reason the authors believe that learning activities involving embodiment lead to a greater chance of retrieval and retention of this information.

In order to implement this idea inside classroom activities different research teams developed different formats. \textit{Mixed Reality} is a term describing the space in between entirely real-world environments and entirely virtual environments  \cite{milgram:taxonomy} where computational devices are used in order to augment the physical reality, in other words activities that are conducted in the physical space of the class with the help of computer simulations and different interfaces for interacting with it. Mixed Reality can take a number of forms such digital overlays on real world scenes, for example projections on walls, floors and other surfaces. Evidence suggests that this format supports productive social interactions and a generates a positive atmosphere of playfulness and experimentation. A very important concept introduced in this framework is the \textit{body-based metaphors} that enable to translate natural movements of the user body into a behavior of virtual elements \cite{lindgren:supporting}. If the metaphor holds than the simulation responds in a way that is consistent with the child’s intuitive movements and gives feedback that enable immediate learning by doing.

Many elements of Mixed Reality can be found in the work made by Colella \cite{colella:virus} describing \textit{Participatory Simulations} as a way for plunging learners into life-sized, computer-supported simulations where the user acts as a component (an actor) of the simulation. \textit{Simulated investigations} on the other end are developed with the same principles of \textit{Mixed Reality}, but this time the user plays the role of the scientist developing theories about dynamic systems. The student is the observer (doesn't take part in the simulation) and the technology play a supporting role for the scientific research that is conducted. \textit{Embedded Phenomena} is a learning technology framework in which simulated scientific phenomena are mapped onto the physical space of classrooms \cite{moher:embedded} and continue over a long period of time: two to three months in which the investigated phenomena evolves overtime.

\subsection{Participatory Simulations}
\textit{Participatory Simulations} is a term introduced by Colella \cite{colella:virus} and describes a new set of activities that immerse the students inside the phenomena itself by transforming them in elements of it (e.g. a bee \cite{peppler:beesim}, an infected individual \cite{colella:virus} and a squirrel foraging for food \cite{gnoli:hunger_games}). Sensor based devices are carried or worn by participants (e.g. \textit{Thinking Tags} \cite{colella:virus}, GPS enabled handheld devices) and they allow exchange of contextually relevant information. Participants experience the simulations in many different ways: at local level (from the perspective of the element that they are impersonating), global level where they can see how actions of other individuals affect patterns within the overall system and from outside (e.g. after the simulation ends, students look at the collected data and try to explain the phenomena). Technology is mainly used in order to support the game with all the rules and the possible interactions. Examples of participatory simulations are: BeeSim \cite{peppler:beesim}, Hunger Games \cite{gnoli:hunger_games}, Virus Simulation \cite{colella:virus} and Savannah \cite{facer:savannah} and they are explained in details in the next chapter.

\subsection{Simulated investigations}
In \textit{Simulated Investigations} the student assumes the role of a scientist and the main focus is on the studied phenomena. The role of the technology is simulating the phenomena, helping the user during the inquiry with tools that resemble the real ones, supporting findings and helping documenting facts and discoveries. Another interesting characteristic is that those simulations last for many weeks in which students can gather and aggregate evidence to answer questions related to the object of inquiry.

\subsection {Embedded Phenomena}
Embedded Phenomena term was created by Moher \cite{moher:embedded} and describes a simulated scientific investigation about certain phenomena that is mapped inside the physical space of the classroom. It intends to create opportunities for students to explore a certain scientific domain with the goal of gathering as many information as they can in order to answer questions and learning the specific lexis about the topic during an extended period of observation (that usually is few months). The fundamental element of every embedded phenomena is the mapping between different part of the phenomena inside the physical space of the classroom. The second element is the state of the simulation that is represented though different media, distributed around the classroom representing portals into that phenomenon. The simulations are persistent and their state is observable in every moment during the whole period of time. The state can also be manipulated though different interfaces and the students can use manipulations for collectively gather evidence to answer a questions and solve problems. 

This framework represents the perfect opportunity for testing how objects became a component of the simulation and what are the characteristics that they must have for producing real learning advantages. In the user study described in \autoref{chap:user_study} tangible objects are integrated inside \textit{WallCology} Embedded Phenomena \cite{moher:wallcology}.

\subsection{Macroworlds}

A successive evolution of Embedded Phenomena are the Macroworlds that are technology-enhanced learning environments that leverage the notions of ubiquitous computing introduced by Weiser \cite{weiser:computer} and the distributed interaction \cite{luyten:distributed} to provide engaging ways for students to experience and interact with classroom sized scientific phenomena. Fundamental capability for a \textit{Macroworld} is the tracking of object movements inside a room in order to use it as an interface between the user and the simulation that is running inside the computing environment.


\section{Tangible interaction design}
The second main trend of research on embodiment is focused on \textit{Tangible User Interfaces (TUI)} also called \textit{Graspable User Interfaces}. They are a kind of interface where the user manipulates physical objects in order to interact with a digital environment. They are different from the graphical user interfaces in many aspects, the most important one is that different physical and familiar manipulations are being closely coupled with different kinds of digital information changes. Learning Science community has progressively gained interest in this kind of interfaces. In this section I want to provide an overview on how they are used and the results of previously researches on them.

Empirical studies conducted by Marshall \cite{marshall:tangible} suggested that there's a close link between physical activity and cognition, for this reason he supports the \textit{learning benefits of physicality}. The second and very important argument is that tangibles improve collaboration though the usage of \textit{Shareable Interfaces} that are technologies specifically designed to enable groups to work on shared representations. Primary school activities are mainly developed with the concept of whole class in mind: very often require children to collaborate and to use social interaction in order to achieve a goal. Young children activities are designed to be social and develop a sense of community and for this reason interfaces supporting collaboration are particularly suited for supporting those kind of activities. Well designed tangibles encourage more equitable participation from group members. \textit{Accessibility}, intended as intuitiveness during the usage of an interface, is another advantage: manipulating a toy is more intuitive than moving a cursor on the screen as it happens during \textit{Graphical User Interface} usage. The last advantage is that physical artifacts can increase the \textit{playfulness of learning}. Tangibles are well suited for supporting two kind of learning activities \cite{marshall:tangible}: \textit{exploratory activities} and \textit{expressive activities}. The first involves the learner that explores an existing representation or model of a topic, typically based on the ideas of a teacher or domain expert. Expressive activities are based on the external representation of a domain that can include system generated representation created using the data that users are collecting and inserting.

Another argument that supports the introduction of tangibles in learning science applications is coming from the research conducted by Bakker \cite{bakker:moso}. In this study tangibles are used in order to manipulate sounds: objects with different shapes were connected to properties of sounds like frequency, amplitude and tempo. Students can explore and intuitively understand those properties manipulating the tangibles. The main finding of this research is that the physicality of the tangibles handles to reason about the targeted abstract concepts though the usage of \textit {embodied metaphors}: unconscious knowledge originating in body movement that can be applied automatically. In a second study \cite{bakker:embodied} also states that "interaction models based on embodied knowledge can support children’s learning in abstract domains." \cite{bakker:embodied} According with the author, people unconsciously apply embodied schemata introduced by manipulating physical objects reasoning and talking about abstract concepts. Physical objects play a major role in bridging the abstract and the concrete. This is another point that supports the introduction of tangibles interfaces: the creation of embodied schemata for reasoning about abstract concepts. 

Dillenbourg research \cite{dillenbourg:you} about collaborative learning demonstrated that collaboration and cooperation are able to improve learning. An interesting technology used for implementing collaborative learning is the \textit{ThinkerLamp} \cite{do:tinkerlamp}: a tangible tabletop learning environment explicitly designed to support classroom orchestration. In this environment little shelf reproductions are used as tangible interface in order to insert shelf position in a warehouse planning system simulation. A down-projected video is used as output in order to display the outcome of the simulation. Every tangible has a QR-code on top of it that is tracked by a computer vision system able to recognize the position and the orientation of every tangible and communicate it to the simulator. The results of the research are that tangibles support class awareness and facilitate both group and class-wise discussions.

Price study \cite{price:snark} about playful learning uses tangibles in order to interact with a virtual creature called Snark enabling exploring though interacting that take advantage of the exploration capability offered by tangible interfaces that I mentioned before. The goal of the research is "to investigate how various kinds of novel tangibles support playful learning." \cite{price:snark} The results of the study claim another time that interactions with tangibles encourage collaboration but also highlight the "potential for news ways of interacting, exploring and learning." \cite{price:snark}

\section{Improving school activities with embodiment}
In the previous sections I described \textit{whole body interaction} and \textit{tangible interaction} and cited researches in the field highlighting the results. In this section I want to take a step back and give a general motivation and answering to the question: how can the introduction of embodiment improve classroom activities? With embodiment I mean the usage of \textit{whole body movements} (that can be defined also as \textit{ambulatory interaction}) and \textit{tangible user interfaces} (denoted with \textit{tangibles} from this point on).

As already pointed out in Lindgren research \cite{lindgren:emboldened} there's a strong relation between physical actions and concepts that are being learned. It is also claimed in a research conducted with Abrahamson that "every school subject is embodied, even the mathematical subjects: fundamental STEM knowledge is itself shaped by the embodied nature of the human mind." \cite{abrahamson:embodiment} In order to support this statement the authors created a series of systems like the desktop application \textit{Mathematical Imagery Trainer} and the \textit{Mixed Reality} environment \textit{MEteor} that improve the intuitive understanding of the physical laws governing objects movement in space. Antle research supports this concept and also assert that \textit{embodied metaphors} improve usability of a system and improve learning of the underlying concept as long as other interacting factor properties are present: "discoverability, perceivability of feedback and duplicity of structural isomorphism." \cite{antle:body}

In order to improve classroom activities conducted by young children in primary school, a fundamental characteristic of the embodied interaction, maybe the most important, is that it creates opportunities for social interaction. Primary school activities, as already pointed out in the previous section, are mainly centered on groups of students or the whole class with the goal of developing social skills of the young students and allow them to learn from each other. Embodied interaction is well known for creating opportunities for social interaction. Moher \cite{moher:embedded} claims that in \textit{Embedded Phenomena} applications (where embodiment concepts are widely used) like \textit{Hunger Games} \cite{gnoli:hunger_games} are at the base of new unanticipated phenomena that appear thanks to social learning activities and it become a "source for the development of new understandings" \cite{gnoli:hunger_games} about the scientific topic.

Embodiment can leverage social protocols more than traditional graphical user interfaces, and an explanation can be found in the research published by Horn \cite{horn:role}. The author claims that tangibles "activates intricate patterns of social activity. With these patterns of activity come associated physical, cognitive, and emotional resources that individuals apply to the situation." \cite{horn:role} Children are placed in a situation where they can leverage all their social protocols learned and used up to that moment. Actions like coordinating the usage of shared tangibles are automatically associated with "historically elaborated social constructions and conventions" \cite{horn:role} that describe how to access those resources without the need to learn or develop new ways of coordinating. An important discovering of this research is the fact that tangibles are seen as \textit{Cultural Artifacts}: users treat them accordingly with the idea suggested by their look and feel. If they look like familiar objects they will be used in the same way of those objects: a rope with handles that afford grasping will be used as a jump rope in contrast with a rope without handles that can suggest other interactions with it. Horn also point out that "peers teach each other strategic approaches to play that go well beyond the rules of the game itself" \cite{horn:role} and in this way "kids teach and learn from each other through play" \cite{horn:role} using tangibles.

The last argument that I want to mention in favor of embodiment is the ability to engage and enthusiasm young students creating a strong positive feeling and connection between them and the subject learned. This argument is supported by a large number of researches. Despite the fact that is not easy to demonstrate the connection between engagement and improving learning skills, it is possible to assert that "fun and enjoyment are well known to be effective in children’s development" \cite{Clements:playing} "both supporting and deepening learning" \cite{kafai:constructionism}. As Colella demonstrated in her research \cite{colella:participatory}, students remain engaged (engagement increases attention to the activity) also after the simulation ends: during the analysis of the results. It is extremely positive for learning because this is the phase when intuitive understanding learned during the experimentation is converted into persistent knowledge. Also Price \cite{price:snark} supports this claim adding that tangibles must be specifically created for "stimulating engagement, reflection and understanding" \cite{price:snark} and in this way improving the motivation in doing the activity. An interesting suggestion for creating engaging, playful and pleasurable environments comes from this research: "the key is to be able to design tangible arrangements by which the known and familiar can be recombined in new and unfamiliar ways." \cite{price:snark} In addition to fun, the author believes "that playful learning should include the core inter-related learning activities: \textit{exploration through interaction}, \textit{engagement}, \textit{reflection}, \textit{imagination, creativity and thinking at different levels of abstraction}, \textit{collaboration}." \cite{price:snark} Engagement can also be used for achieving other goals than learning, as demonstrated in the research by Pares \cite{soler:novel}. In this research the author uses an inflatable slide with an overlay of Virtual Reality obtained projecting a video and using children movements as input "for developing full-body, multi-user interactive experiences in which children can have rich physical and social activity." \cite{soler:novel} The goal is "countering lack of physical activity and lack of socialization in children" \cite{soler:novel} that often happen in the developed societies.

\section{Sharing interfaces and whole classroom activities}
Primary school classroom activities, as already introduced in the previous section, are often centered on group collaboration for solving problems. The computer technology used in our day to day life is not well suited for those activities, for example personal computers have one mouse and students have to build ways for sharing the media. Many studies were conducted in order to discover what is the best way of building shared interfaces and what are the main properties that must be taken into account during the design.

Studies on collaborative interactions in shared environments conducted by Inkpen \cite{inkpen:turn} and Myers \cite{myers:floor} both examined how different multi-user interactions impact learners’ engagement and participation to didactic activities. The studies are motivated by the intuition that face-to-face collaboration with classmates and usage of shared media during paper based activities is an important part of students daily routine. The first study outcome is that "children exhibit collaborative behavior similar to their interactions during paper-based activities"  \cite{inkpen:turn}. Authors also claim that "providing children with technology that supports concurrent, multi–user interaction can positively impact their engagement, participation, and enjoyment of the activity" \cite{inkpen:turn}. Another remarkable result of this study is that computer supported activities that lack in physicality may decrease the effectiveness of the collaboration, the user performance and the motivation of the students. The second study compares turn taking protocols (called "floor control") that allow groups of students to complete jigsaw puzzles. The result obtained comparing 8 different collaborative environments is that giving every user a separate cursor (Multi-Cursor parallel condition) works best.

While designing activities it is very important to have in mind the whole class, as claimed in researches conducted by Slotta \cite{slotta:orchestrating}. In the theory of class orchestrating it is very important to use technology for supporting those kind of activities, the "script is \textit{orchestrated} by the teacher, who in turn is enabled by a smart classroom learning environment, that includes ambient displays and spatial mapping of activities." \cite{slotta:orchestrating} According with Slotta this learning environments must be seen as \textit{Knowledge Community and Inquiry} (KCI) where "individuals are scaffolded in all aspects of the curriculum, as they engage in reflection, critique, discussion, or design activities performed individually, in groups, or as a wider learning community." \cite{slotta:orchestrating} The KCI reflect a broader behavior: we learn as a society, not as individuals, and for this reason activities that take into account the entire class are better compared with activities that involve only a small number of students. A confirmation of this theory emerges form the \textit{National Survey of Science and Mathematics education} \cite{fulp:national} where is stated that 67\% of the primary school students work in groups at least once a week, furthermore it is also stated that "elementary school science teachers expressed a need for help in using instructional technology." \cite{fulp:national}

\section{How to better support scientific science inquiry}
In this chapter I described a series of studies that independently confirmed the following characteristics of primary school classroom activities:
\begin{itemize}
    \item Embodied interaction can be employed for improving engagement and enjoyment and let students focus more on the activity while improving social skills and counter the lack of physical activity.
    \item Tangible user interfaces can be leveraged in a number of different ways.\\ The most relevant one is that creates opportunities for productive disciplinary discourses about the scientific topic and take advantage of already learned sharing social protocols.
    \item Sharing interfaces are very good for developing social skills and leads to discussions about the actions that must be taken during inquiries for obtaining the disired result.
    \item They must take into account the whole classroom.
\end{itemize}
In this thesis I take into account all those characteristics and I design and implement a system called RoomPlaces that is the first step in creating a general purpose system ready out of the shelf able to support the activities that will be created for the classroom of the future. Those activities will employ many tangible artifacts interconnected, for this reason I use the term \textit{Classroom of Things}.

There are two motivations for introducing tangibles in curricular activities, especially in scientific inquiries. The first motivation abundantly supported by the cited literature is that children will enjoy more the work done using tangible interfaces and they will be more motivated and focused on the task. The second motivation is that creates productive discourses and promotes information sharing between individuals. This motivation is more speculative and with this thesis I provide evidence to support it.

There are also motivations against the introduction of tangibles. The main one is that they slow down the work of the students requiring more time for accomplish the same results. I extensively discuss this topic in \autoref{chap:user_study} reporting that some student agrees with this motivation, but many more argue against it accepting the trade off. The last motivation is the difficulty that a teacher can have in controlling the class while using those tangibles and also for this I remand the discussion to \autoref{chap:user_study}.

In the \autoref{chap:lit_rev} I describe many systems found in literature using embodied interaction that already powered activities in the last decade. For every one of them I highlight what are the interesting characteristic that can be combined in a general system with the goal of supporting tangibles and whole body interaction.

In the \autoref{chap:room_places} I describe RoomPlaces system: what are its characteristics, specifications, design choices and how I implemented it.

In the \autoref{chap:user_study} I describe an entire user study conducted during the fall semester 2015 at a primary school in Chicago where WallCology is run continuously for two months. I report all the development phases with a specific focus on the parts that use RoomPlaces for interacting with tangibles. This user study serves as a testing ground for demonstrating that RoomPlaces has indeed the ability of supporting real cases of science inquiry and that it incredibly minimizes the time spent during development. It is also incorporate interesting research questions that are unanswered until now about shared control tangibles: 
% Describe

\begin{itemize}
\item How do teachers and learners coordinate their use of shared tangibles?
\item Can the use of tangibles motivates the students? Can the need for coordination of shared tangibles create opportunities for productive disciplinary engagement? (Engle and Conant \cite{engle:guiding}) Do they become cultural artifacts (a la Horn \cite{horn:role})?
\end{itemize}
