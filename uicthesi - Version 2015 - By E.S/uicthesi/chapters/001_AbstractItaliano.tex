\abstract

\label{chap:abstract_italiano}

Questa tesi presenta due contributi distinti e complementari. Il primo contributo é uno strumento software chiamato RoomPlaces che fornisce un sistema di localizzazione di prossimità che sfrutta la tecnologia Bluetooth Low Energy. Consente agli sviluppatori di creare applicazioni che hanno accesso alla posizione di un oggetto o una persona all’interno della classe e fornisce notifiche quando questi arrivano o lasciano determinate aree. Queste capacità possono essere utilizzate per creare una serie di servizi come appello automatizzato, gestione dell’inventario e monitoraggio dell’utilizzo delle risorse. Questa tesi é focalizzata su come utilizzare la posizione di oggetti come interfaccia verso simulazioni scientifiche didattiche svolte da studenti di scuola primaria.  

Il secondo contributo è uno studio esplorativo riguardante l’utilizzo di artefatti tangibili a scopo didattico condotto utilizzando RoomPlaces per creare WallCology (Moher, et al., 2008). La posizione di artefatti tangibili è utilizzata per controllare una simulazione distribuita di dinamica degli ecosistemi. Durante questa attività gli studenti lavorano in gruppi con l’obiettivo di scoprire le interazioni presenti tra specie di creature che vivono negli ecosistemi virtuali. Gli studenti hanno la possibilità di modificare la quantità di creature presenti e di introdurre o rimuovere specie. Gli ecosistemi sono inoltre programmati per subire interferenze dall’esterno come per esempio l’aumento della temperatura, la distruzione di parti dell’habitat e l’invasione di specie esterne.

Studi riguardanti la costruzione di conoscenza all’interno di comunità mostrano una notevole difficoltà per gli studenti nel condividere le informazioni tra di loro. Lo studio esplorativo é stato effettuato su studenti di scuola primaria utilizzando un solo gruppo di artefatti contenenti localizzatori Bluetooth LE, creando un unico punto di aggregazione con l’intento di incentivare lo scambio di informazioni tra gruppi. I risultati ottenuti evidenziano i benefici degli artefatti tangibili: aumentano la motivazione a partecipare nelle attività didattiche, creano opportunità per discussioni legate alla simulazione e aumentano la percezione di autonomia e controllo durante l’attività.
