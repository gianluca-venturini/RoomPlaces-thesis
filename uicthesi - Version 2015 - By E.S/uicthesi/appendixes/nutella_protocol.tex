\chapter{Nutella protocol}

\label{apdx:nutella_protocol}

In this appendix I describe how nutella protocol works and how it implements the functionalities requested for providing push and pull strategies.

Nutella protocol encapsulate messages into MQTT protocol that allows binary content. By design nutella protocol allows only content that can be encoded as JSON (basically everything) and add a little payload that allows both end to communicate using the right strategy.

A nutella message looks like this:

\begin{lstlisting}[language=NutellaProtocol]
{
    "id": "req_res_id",
    "from": {
        "type": "run",
        "run_id": "my_run_id",
        "app_id": "my_app_id",
        "component_id": "my_component_id",
        "resource_id": "my_resource_id"
    },
    "type": "request",
    "payload": {
        /* Custom data */
    }
}
\end{lstlisting}

In order to distinguish between push and pull strategy it is used "type" field (the more external one) that can be "request", "response" or "publish".
\begin{itemize}
    \item \textit{Request}: from client to server
    \item \textit{Response}: from server to client
    \item \textit{Publish}: peer communication
\end{itemize}

The custom data in inserted in the "payload" field and is encoded in JSON format.

The "from" field is passed as is to the subscribe and handle request callbacks and it is used mainly for debugging purposes.

This protocol is never seen from the developer because the nutella.net primitives automatically incapsulate and decapsulate the payload automatically when they're called.

The primitives are:
\begin{itemize}
    \item \textit{nutella.net.subscribe(channel, callback(message, from))}
    \item \textit{nutella.net.publish(channel, message)}
    \item \textit{nutella.net.request(channel, request)}
    \item \textit{nutella.net.handle\_requests(channel, callback(request, from))}
\end{itemize}