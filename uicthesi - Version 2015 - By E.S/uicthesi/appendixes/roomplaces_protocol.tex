\chapter{RoomPlaces protocol}

\label{apdx:roomplaces_protocol}


RoomPlaces protocol is a set of rules that allow every RoomPlaces component to communicate. It is specifically designed for supporting the following functionalities:
\begin{itemize}
    \item Add resource
    \item Delete resource
    \item Update the resource position for automatic tracked resources
    \item Update the resource location for static resources
    \item Update the state of the resource (data associated to that resource)
    \item Add, modify and remove a discrete tracking coordinate system
    \item Notify arrival and departure events (when a dynamic resource is near a static resource)
\end{itemize}

The protocol supports also groupings: a group of resources allow more efficient resource retrieval. This functionality was not implemented in the first prototype.

The majority of the functionalities can be accessed in both \textit{push} (subscribing to a channel and wait for update) and \textit{pull} (explicitly requesting the data) modality. The idea behind this choice is that \textit{nutella.location} needs the state of the entire system during the start-up phase and then only small updates for being synchronized with the central server.

In this appendix I describe the RoomPlaces protocol using BNF \cite{naur:revised} for the grammar. This protocol is based on the notion of nutella channels, every channel has also a modality that can be publish/subscribe or request/response (a channel can also support both the modalities because nutella protocol support it \autoref{apdx:nutella_protocol})


\section{Publish/Subscribe channels}
All the channels are padded with \textit{/location} that is the convention name for \textit{nutella.location} communications.

\begin{table}
\centering
\begin{tabular}{ | l | l | l | l | }
    \hline Channel & Description & Recipient & Message \\
    \hline /resource/add                & Add a new resource        & Server & $<$resource\_add$>$ \\
    \hline /resource/remove             & Remove a resource         & Server & $<$resource\_remove$>$ \\
    \hline /resource/update             & Update a resource         & Server & $<$resource\_update$>$ \\
    \hline /resources/update            & Update a set of resources & Server & $<$resources\_update$>$ \\
    \hline /resources/added             & Set of resources added    & Client & $<$resources\_added$>$ \\
    \hline /resources/removed           & Set of resources removed  & Client & $<$resources\_removed$>$ \\
    \hline /resources/updated           & Set of resources updated  & Client & $<$resources\_updated$>$ \\
    \hline /room/update                 & Update the room           & Server & $<$room\_update$>$ \\
    \hline /room/updated                & Notify a room update      & Client & $<$room\_updated$>$ \\
    \hline /resource/static/$<$rid$>$/enter & Notify an enter event & Client & $<$event\_enter$>$ \\
    \hline /resource/static/$<$rid$>$/exit  & Notify an exit event  & Client & $<$event\_exit$>$ \\
    \hline /tracking/discrete/update    & Update the discrete coord & Server & $<$discrete\_update$>$ \\
    \hline /tracking/discrete/updated   & Notify disc coord update  & Client & $<$discrete\_updated$>$ \\
    \hline
\end{tabular}
\end{table}

\section{Request/Response channels}
All the channels are padded with \textit{/location} that is the convention name for \textit{nutella.location} communications.

Every request is done from the clients and the server will send a response. 
\begin{table}
\centering
\begin{tabular}{ | l | l | l | l | }
    \hline Channel & Description & Response \\
    \hline /resources                & Set of resources                 & $<$resources\_response$>$ \\
    \hline /room                     & Room Information                 & $<$room\_response$>$ \\
    \hline /tracking/discrete        & Discrete coordinate information  & $<$discrete\_response$>$ \\
    \hline
\end{tabular}
\end{table}

\section{Protocol grammar}
The RoomPlaces protocol grammar is a subset of JSON grammar because it uses JSON primitives (present in the majority of the programming languages) for "stringifying" objects.

\begin{lstlisting}[language={}]
<resource_add> ::= {"rid": "<rid>", "model": "<model>", "type": "<type>" [, "proximity_range": <float>]}
<resource_remove> ::= {"rid": "<rid>"}
<resource_update> ::= {"rid": "<rid>", (<continuous> | <discrete> | <proximity> | <parameters>)}
<resources_update> ::= {"resources": `[' [<resource_update> (, <resource_update>)* ] `]'}
<resources_added> ::= {"resources": `[' <resource> (, <resource>)* `]'}
<resources_removed> ::= 
<resources_updated> ::= {"rid": "<rid>", "model": "<model>", "type": "<type>", (<continuous> | <discrete> | <proximity>), <parameters_updated> [, "proximity_range": <float>]}
<room_update> ::= {"x": <float>, "y": <float>}
<room_updated> ::= <room_update>
<event_enter> ::= {"resources": `[' <resource> (, <resource>)* `]'}
<event_exit> ::= {"resources": `[' <resource> (, <resource>)* `]'}
<discrete_update> ::= {"tracking": <discrete_tracking>}
<discrete_updated> ::= <discrete_update>

<resources_response> ::= {"resources": `[' <resource> (, <resource>)* `]'}
<room_response> ::= <room_update>
<discrete_response> ::= {tracking: <discrete_tracking>}

<resource> ::= {"rid": "<rid>", "model": "<model>", "type": "<type>"}
<continuous> ::= "continuous": {x: <float>, y: <float>}
<discrete> ::= "discrete": {"x": <discrete_n>, "y": <discrete_n>}
<proximity> ::= "proximity": {["rid": "<rid>", "distance": <float>]}
<parameters> ::= "parameters": `['[<parameter> (, <parameter>)*]`]'
<parameter> ::= {"key": "<string>" , ("value": "<string>" | "delete": true)}
<parameters_updated> ::= "parameters": {(<key>: '<string>')*}
<discrete_tracking> ::= {"x": <float>, "y": <float>, "width": <float>, "height": <float>, "n_x": <float>, "n_y": <float>, "t_x": "<discrete_type>", "t_y": "<discrete_type>"}
<discrete_type> ::= NUMBER | LETTER
<discrete_n> ::= <int> | <uppercase_char>
<key> ::= <string>
<rid> ::= <string>
<int> ::= <digit>+
<string> ::= (<digit> | <uppercase_char> | <lowercase_char>)*
<digit> ::= 0 | 1 | 2 | 3 | 4 | 5 | 6 | 7 | 8 | 9
<uppercase_char> ::= A | B | C | D | E | F | G | H | I | J | K | L | M | N | O | P | Q | R | S | T | U | V | W | X | Y | Z
<lowercase_char> ::= a | b | c | d | e | f | h | h | i | j | k | l | m | n | o | p | q | r | s | t | u | v | w | x | y | z
\end{lstlisting}



